%%%%%%%%%%%%%%%%%%%%%%%%%%%%%%%%%%%%%%%%%
% Medium Length Professional CV
% LaTeX Template
% Version 2.0 (8/5/13)
%
% This template has been downloaded from:
% http://www.LaTeXTemplates.com
%
% Original author:
% Trey Hunner (http://www.treyhunner.com/)
%
% Important note:
% This template requires the resume.cls file to be in the same directory as the
% .tex file. The resume.cls file provides the resume style used for structuring the
% document.
%
%%%%%%%%%%%%%%%%%%%%%%%%%%%%%%%%%%%%%%%%%

%----------------------------------------------------------------------------------------
%	PACKAGES AND OTHER DOCUMENT CONFIGURATIONS
%----------------------------------------------------------------------------------------

\documentclass{resume} % Use the custom resume.cls style
\usepackage{hyperref, textcomp, lipsum}
\usepackage{enumitem}
\hypersetup{
    colorlinks=true,
    linkcolor=blue,
    filecolor=blue,      
    urlcolor=blue,
}
\usepackage[left=0.5in,top=0.4in,right=0.5in,bottom=0.4in]{geometry} % 

\name{Kalpesh Krishna}
\address{\textit{\href{mailto:kalpeshk2011@gmail.com}{kalpeshk2011@gmail.com} \\ \href{https://www.linkedin.com/in/kalpesh-krishna-6b3827a6/}{LinkedIn} \\ \href{https://github.com/martiansideofthemoon}{Github} \\ \href{http://martiansideofthemoon.github.io/}{Webpage}\footnotemark[1]}} % Your phone number and email

\begin{document}
\footnotetext[1]{Use URL \href{http://martiansideofthemoon.github.io}{martiansideofthemoon.github.io} in case hyperlinks don't work}
%----------------------------------------------------------------------------------------
%	EDUCATION SECTION
%----------------------------------------------------------------------------------------
\vspace*{-1mm}
\begin{rSection}{Education}
{\bf University of Massachusetts, Amherst} \hfill {Sept '18 - Present} \\ \textit{MS/PhD} in Computer Science (advised by \textit{\href{https://people.cs.umass.edu/~miyyer/}{Prof. Mohit Iyyer}}) \\\\
{\bf Indian Institute of Technology Bombay}, \textit{Mumbai, India} \hfill {July '14 - July '18} \\ 
\textit{B.Tech} in Electrical Engineering (\textit{Minor} in Computer Science \& Engineering)\\
Major GPA: \textbf{9.74/10} (\textit{$\mathbf{2^{nd}}$ among 66 students}) (Minor GPA: 10/10)\\
\textbf{Sharad Maloo Memorial Gold Medalist} for outstanding all-round excellence. \\\\
Thesis: \textbf{Constraint-Driven Learning in NLP Applications} (under \textit{\href{https://www.cse.iitb.ac.in/~pjyothi/}{Preethi Jyothi}}) \\
Conducted a literature survey on machine learning models utilizing posterior distribution constraints, in the context of Part-of-Speech tagging, named-entity recognition and sentiment classification. Successfully implemented and matched the results of a popular \href{https://en.wikipedia.org/wiki/Convolutional_neural_network}{CNN}-based sentiment classification \href{https://arxiv.org/abs/1408.5882}{baseline}. Currently working on extending this architecture using a ``student-teacher'' distillation framework using \href{http://www.cs.columbia.edu/~mcollins/papers/tagperc.pdf}{perceptron} / \href{http://jmlr.csail.mit.edu/papers/volume7/crammer06a/crammer06a.pdf}{PA} algorithms, inspired by CMU's \href{https://arxiv.org/pdf/1603.06318.pdf}{Harnessing Deep Neural Networks with Logic Rules}.
\end{rSection}
%----------------------------------------------------------------------------------------
%	WORK EXPERIENCE SECTION
%----------------------------------------------------------------------------------------
\vspace*{-1.5mm}
\begin{rSection}{Publications}
\begin{itemize}[leftmargin=*]
\item \textbf{K. Krishna}, P. Jyothi \& M. Iyyer\\\textit{Revisiting the Importance of Encoding Logic Rules in Sentiment Classification}\\ \textbf{EMNLP 2018} (\textit{oral presentation, short paper}) \\
\texttt{arXiv:1808.07733 [cs.CL]}
\item \textbf{K. Krishna}, S. Toshniwal \& K. Livescu\\\textit{Hierarchical Multitask Learning for CTC-based Speech Recognition}\\ (\textit{submitted to \textbf{SLT 2018}}) \\ \texttt{arXiv:1807.06234 [cs.CL]} 
\item \textbf{K. Krishna}, L. Lu, K. Gimpel \&  K. Livescu\\\textit{A Study of All-Convolutional Encoders for Connectionist Temporal Classification}\\ \textbf{ICASSP 2018} (\textit{Awarded SPS Travel Grant}) \\ \texttt{arXiv:1710.10398v2 [cs.CL]} 
% \item \textbf{K. Krishna} \& P. Jyothi - \textit{Leveraging n-grams in RNN Language Models via Loss Functions} (previously submitted to EMNLP - 2017)
\end{itemize}
\end{rSection}
\vspace*{-1.5mm}
\begin{rSection}{Experience}
\begin{rSubsection}{Toyota Technological Institute at Chicago}{May '17 - July '17}{Visiting Student under \href{http://ttic.uchicago.edu/~klivescu/}{Karen Livescu}, \href{http://ttic.uchicago.edu/~llu/}{Liang Lu} and \href{http://ttic.uchicago.edu/~kgimpel/}{Kevin Gimpel}}{Chicago, IL}
\item Designed, implemented (in TensorFlow) and analyzed \href{https://en.wikipedia.org/wiki/Connectionist_temporal_classification_(CTC)}{CTC}-based end-to-end pure 1-D and 2-D ResNet-style \href{https://en.wikipedia.org/wiki/Convolutional_neural_network}{CNN} architectures to speed up character-based conversational Speech Recognition systems. Implemented the lexicon-free language model beam-search decoding \href{http://deeplearning.stanford.edu/lexfree/lexfree.pdf}{algorithm}. Achieved \textbf{2.5x better} training time, \textbf{16x better} decoding time and competitive results against LSTM baselines. Built web tools in \href{http://flask.pocoo.org/docs/0.12/}{Flask} to increase research \& engineering productivity. Submitted a paper to ICASSP-2018.
\end{rSubsection}
\vspace*{-3mm}
\begin{rSubsection}{Mozilla Foundation}{May '16 - August '16}{Google Summer of Code student under \href{https://github.com/armenzg}{Armen Zambrano}}{Mumbai, India / London, UK}
\item Selected for the prestigious \href{https://en.wikipedia.org/wiki/Google_Summer_of_Code}{Google Summer of Code} program ($16\%$ proposals selected in 2016) to work on Mozilla's Continuous Integration framework. Collaborated with multiple teams to fix deficiencies in \href{https://en.wikipedia.org/wiki/Firefox}{Firefox}'s testing dashboard \href{https://wiki.mozilla.org/EngineeringProductivity/Projects/Treeherder}{Treeherder} and task-execution framework \href{https://docs.taskcluster.net/}{TaskCluster}. Wrote the first version of \href{https://docs.taskcluster.net/manual/using/actions}{Action Tasks}, optimizing it using basic graph algorithms. Attended Mozilla's conference at London to discuss Action Tasks, Mozilla's automation and \href{https://wiki.mozilla.org/Auto-tools/New_Contributor/Quarter_of_Contribution}{Quarter of Contribution}.
\end{rSubsection}
\end{rSection}
\vspace*{-1.5mm}
\begin{rSection}{Scholastic Achievements}
\begin{itemize}[leftmargin=*]
\itemsep -0.5em 
\item Received the Institute Academic Prize for standing $2^{nd}$ in the  sophomore year 2015-16.
\item Awarded AP grade (Top 1\% of class) in \textit{Computer Programming}, \textit{Basic Biology} and \textit{Data Analysis}.
\item Selected for \href{https://www.lti.cs.cmu.edu/2017-jsalt-undergraduate}{JSALT '17}, organized by JHU's \href{https://www.clsp.jhu.edu/}{Center for Language and Speech Processing}\footnotemark[2]. \footnotetext[2]{Couldn't attend due to clashing college schedule}
\item \textbf{Top 10} at the Astronomy Olympiad's Indian Selection Camp for IOAA '14, ($\sim$20000 applicants).
\item All India Rank 2 \textit{(out of 132k)} in \href{https://en.wikipedia.org/wiki/Indian_Certificate_of_Secondary_Education}{ICSE} '12, All India Rank 93 in \href{https://en.wikipedia.org/wiki/Joint_Entrance_Examination}{JEE Advanced} '14 \textit{(out of 126k)} and All India Rank 34 in \href{https://en.wikipedia.org/wiki/Joint_Entrance_Examination}{JEE Mains} '14 \textit{(out of 1.4M candidates)}.
\item Selected for the Kishore Vaigyanik Protsahan Yojana Award '14 (1000 out of 20000 applicants).
\end{itemize}
\end{rSection}

\begin{rSection}{Projects}
\begin{rSubsection}{Neural Language Models}{December '16 - April '17}{R\&D Project under \href{https://www.cse.iitb.ac.in/~pjyothi/}{Preethi Jyothi}}{Computer Science \& Engineering, IIT Bombay}
Implemented (in TensorFlow) and matched the results of popular \href{https://en.wikipedia.org/wiki/Language_model}{language modelling} baselines (with \href{https://en.wikipedia.org/wiki/Treebank}{PTB}). Designed several novel neural language modelling architectures at word, sub-word and character level, aimed at morphologically rich languages. Designed and analyzed a new ``model-mimicking'' loss function which leveraged \href{https://en.wikipedia.org/wiki/N-gram}{$n$-gram} statistics. Conducted experiments comparing the role of stochastic optimizers in language modelling. Submitted this research work as a short-paper to EMNLP-2017.
\end{rSubsection}
%\vspace*{-2mm}
\begin{rSubsection}{Macro Actions in Reinforcement Learning}{October '17 - Present}{RL under \href{https://www.cse.iitb.ac.in/~shivaram/}{Shivaram Kalyanakrishnan}}{Computer Science \& Engineering, IIT Bombay}
\item Applied the \href{https://arxiv.org/pdf/1702.06054.pdf}{Fine Grained Action Repetition} framework to the SARSA($\lambda$) algorithm in the \href{https://github.com/LARG/HFO}{Half-Field Offense (Robocup)} problem. Compared its performance with four alternative action repetition SARSA($\lambda$) variants. Exploring relation of action repetition with reduced discount factor for MDPs.
\end{rSubsection}
\begin{rSubsection}{Blind Dehazing}{October '17 - Present}{Digital Image Processing under \href{https://www.cse.iitb.ac.in/~ajitvr/}{Ajit Rajwade}}{Computer Science \& Engineering, IIT Bombay}
\item Implemented a single image dehazing algorithm to recover airlight, depth maps and dehazed images using the \href{https://www.robots.ox.ac.uk/~vgg/rg/papers/hazeremoval.pdf}{Dark Channel Prior} algorithm. Exploiting the relative degradation (due to haze) of recurring patches across different depths in the image, based on \href{http://www.wisdom.weizmann.ac.il/~ybahat/papers/blindDehazing_ICCP2016.pdf}{Blind Dehazing Using Internal Patch Recurrence}.
\end{rSubsection}
%\vspace*{-2mm}
%\vspace*{-2mm}
\begin{rSubsection}{Brittle Fracture Simulation}{January '17 - April '17}{Advanced Graphics under \href{https://www.cse.iitb.ac.in/~paragc/}{Parag Chaudhuri}}{Computer Science \& Engineering, IIT Bombay}
\item Built a physics framework for simulating cracks and fractures in brittle objects using explicit solver algorithms like Forward-Euler and Runge-Kutta-4, based on \href{http://graphics.berkeley.edu/papers/Obrien-GMA-1999-08/}{Graphical Modelling \& Animation of Brittle Fractures}. Visualized this simuation using Paraview and added global illumination using PovRay.
\end{rSubsection}
%\vspace*{-3mm}
\begin{rSubsection}{Mini-Arbitrary Function Generator}{January '17 - April '17}{Electronic Design Lab under \href{https://www.ee.iitb.ac.in/wiki/faculty/shalabh}{Shalabh Gupta}}{Electrical Engineering, IIT Bombay}
\item Designed and implemented a digital circuit (in VHDL) to receive \href{https://en.wikipedia.org/wiki/Universal_asynchronous_receiver-transmitter}{UART} signals via a custom \href{https://www.gnuradio.org/}{GNURadio} module. Users could specify an input signal via GNURadio, which would be sampled and sent to a Altera Max-V \href{https://en.wikipedia.org/wiki/Field-programmable_gate_array}{FPGA} which played out the signal at a fixed sample frequency (upto 5 MHz). Interfaced this with a Texas Instruments transmitter circuit and successfully carried out \href{http://www.rfwireless-world.com/Terminology/BPSK.html}{BPSK} communication.
\end{rSubsection}
%\vspace*{-3mm}
\begin{rSubsection}{Processor Design}{July '16 - November '16}{Microprocessors under \href{https://www.ee.iitb.ac.in/~viren/}{Virendra Singh}}{Electrical Engineering, IIT Bombay}
\item Designed, implemented and simulated (in VHDL) a six-stage pipelined RISC processor and a multi-cycle RISC processor based on the IITB-RISC instruction set. Wrote an assembler for IITB-RISC.
\end{rSubsection}
%\vspace*{-3mm}
\begin{rSubsection}{Pyraminx Solver}{March '15 - April '15}{Computer Programming under \href{https://www.cse.iitb.ac.in/~kavi/}{Kavi Arya}}{Computer Science \& Engineering, IIT Bombay}
\item Built an Android app using blob detection to identify configurations of a \href{https://en.wikipedia.org/wiki/Pyraminx}{Pyraminx}. Implemented a least-move optimal solver module using graph algorithms. Built a front-end interface using Allegro.
\end{rSubsection}
\vspace*{4mm}
\begin{rSubsection}{Mozilla \& Open Source}{September '15 - August '16}{}{Electrical Engineering, IIT Bombay}
\item Contributed to several open source projects for Mozilla (\href{http://martiansideofthemoon.github.io/projects/}{list}). Took part in the 2nd \href{https://wiki.mozilla.org/Auto-tools/New_Contributor/Quarter_of_Contribution/November_2015}{Quarter of Contribution} and built a webapp (\href{https://github.com/mozilla/wptview}{wptview}), to compare automation test results across different Firefox builds, using Google's \href{https://github.com/google/lovefield}{Lovefield} (IndexedDB library). Mentored three new Mozilla contributors.
\end{rSubsection}
%\vspace*{-3mm}
\begin{rSubsection}{Mood Indigo}{October '15 - December '15}{}{}
\item Contributed towards developing the Android app for \href{https://moodi.org/}{Mood Indigo}, Asia’s largest college cultural festival. The app got 4000 installations and rated 4.6 on 5 on the Playstore.
\end{rSubsection}
%\vspace*{-3mm}
\begin{rSubsection}{Pickup (Taxi Sharing Service)}{March '15 - September '15}{}{Electrical Engineering, IIT Bombay}
\item Built \href{https://en.wikipedia.org/wiki/Representational_state_transfer}{RESTful} APIs and designed an ER Model Database using an MVC Framework \href{https://laravel.com/}{Laravel}. Developed efficient algorithms utilizing the Google Directions API for automatic passenger pair-ups.
\end{rSubsection}
\end{rSection}
\vspace*{-1mm}
\begin{rSection}{Relevant Courses}
\begin{itemize}[leftmargin=*]
\itemsep -0.5em 
\item \textbf{Computer Science} - Data Structures \& Algorithms, Computer Networks, Computer Graphics, Advanced Computer Graphics, Digital Image Processing\footnotemark[3], Operating Systems\footnotemark[3], Discrete Structures\footnotemark[4].
\item \textbf{Machine Learning} - Reinforcement Learning\footnotemark[3], Convex Optimization\footnotemark[4],  R\&D Project, Machine Learning (Coursera).
\item \textbf{Electrical Engineering} - Probability \& Random Processes, Data Analysis \& Interpretation, Information Theory\footnotemark[4], Control Systems, Digital Signal Processing, Microprocessors.
\item \textbf{Mathematics} - Applied Real Analysis\footnotemark[3], Multivariable \& Vector Calculus, Linear Algebra, Differential Equations I \& II, Complex Analysis, Matrix Computations.
\footnotetext[3]{Courses taken in Fall 2017}
\footnotetext[4]{Tentative Course for Spring 2018}
%\footnotetext[4]{Tentative course plan for Spring 2018}
\end{itemize}
\end{rSection}
\vspace*{-1mm}
\begin{rSection}{Technical Skills}
\begin{itemize}[leftmargin=*]
\itemsep -0.5em 
\item \textbf{Strong} - Python (with TensorFlow \& OpenCV), C/C++, JavaScript, VHDL
\item \textbf{Familiar} - MATLAB, PHP (Laravel), Arduino, Java (Android)
\item \textbf{Tools} - TensorFlow, Git, Mercurial, Quartus, \LaTeX
\end{itemize}
\end{rSection}
\vspace*{-1mm}

\begin{rSection}{Responsibilities \& Talks}
\begin{itemize}[leftmargin=*]
\itemsep -0.5em 
\item \textbf{Manager, Web and Coding Club} (2016-17) - Lead a team of 14 sophomores, part of one of the biggest college technical clubs in India, to conduct hobbyistic programming \href{http://wncc-iitb.org/wiki/index.php/Event_Resources}{activities} in the institute. Lead the development of a \href{http://wncc-iitb.org/wiki}{wiki}, a programming guide.  Won \textit{Institute Organizational Color 2016-17}.
\item \textbf{Institute Student Mentor} - Mentoring 12 freshmen and 6 sophomores, helping them get accustomed to the institute life. Helping 1 junior undergraduate overcome academic difficulties.
\item \textbf{Teaching Assistant} -\textit{Computer Programming} in Fall '16 and \textit{Linear Algebra} in Spring '17. Conducted a special help session for \textit{Computer Programming} in Fall '17.
\item \textbf{Talks} - \href{http://ttic.uchicago.edu/~klivescu/SLATTIC/}{TTIC's NLP} paper-reading group, various talks on open source contribution at IIT Bombay.
\end{itemize}
\end{rSection}
\vspace*{-1mm}
\begin{rSection}{Extracurricular}
\begin{itemize}[leftmargin=*]
\itemsep -0.5em 
\item \textbf{Cargill Global Scholar 2016-18} - Selected by the \href{https://en.wikipedia.org/wiki/Institute_of_International_Education}{International Institute of Education} and \href{https://en.wikipedia.org/wiki/Cargill}{Cargill} (largest private corporation in USA) for a global leadership \href{https://www.cargillglobalscholars.com/}{program}. Attended a leadership seminar in Amsterdam (August '17) where we presented a case-study on Sustainable Agriculture in India.
\item Times of India, NIE \textbf{Student of the Year} '11 for all round performance.
\item \textbf{Karate} - Black Belt (1st Dan) trained in \href{https://en.wikipedia.org/wiki/Kyokushin}{Kyokushin Kai} for seven years. Winner at District level.
\item \textbf{Abacus \& Mental Arithmetic} - \href{http://alohama.com/}{Aloha} Grand Master. Winner at National and State level.
\item Stood \textbf{2nd} (as part of a 4-person team) at the Microsoft code.fun.do Hackathon ’15.
\item I enjoy cycling, \href{http://martiansideofthemoon.github.io/archive.html}{blogging}, StackOverflow \href{https://stackoverflow.com/users/5080995/martianwars}{contribution}, star gazing and collecting Rubik's puzzles.
\end{itemize}
\end{rSection}
\begin{rSection}{References}
\begin{center}
\begin{tabular}{cc}
\textbf{Preethi Jyothi} & \textbf{Karen Livescu} \\
Assistant Professor & Associate Professor \\
Computer Science \& Engineering, IIT Bombay & Toyota Technological Institute at Chicago \\
\textit{\href{https://www.cse.iitb.ac.in/~pjyothi/}{webpage} $\diamond$ \href{mailto:pjyothi@cse.iitb.ac.in}{email}}  & \textit{\href{http://ttic.uchicago.edu/~klivescu}{webpage} $\diamond$ \href{mailto:klivescu@ttic.edu}{email}}\\
\\
\textbf{Liang Lu} & \textbf{Kevin Gimpel} \\
Senior Applied Scientist & Assistant Professor \\
Microsoft, Bellevue & Toyota Technological Institute at Chicago \\
\textit{\href{http://ttic.uchicago.edu/~llu/}{webpage} $\diamond$ \href{mailto:llu@ttic.edu}{email}}  & \textit{\href{http://ttic.uchicago.edu/~kgimpel}{webpage} $\diamond$ \href{mailto:kgimpel@ttic.edu}{email}}\\
\\

\end{tabular}
\end{center}
\vspace*{-1mm}
\end{rSection}

%----------------------------------------------------------------------------------------

\end{document}
