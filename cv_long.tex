%%%%%%%%%%%%%%%%%%%%%%%%%%%%%%%%%%%%%%%%%
% Medium Length Professional CV
% LaTeX Template
% Version 2.0 (8/5/13)
%
% This template has been downloaded from:
% http://www.LaTeXTemplates.com
%
% Original author:
% Trey Hunner (http://www.treyhunner.com/)
%
% Important note:
% This template requires the resume.cls file to be in the same directory as the
% .tex file. The resume.cls file provides the resume style used for structuring the
% document.
%
%%%%%%%%%%%%%%%%%%%%%%%%%%%%%%%%%%%%%%%%%

%----------------------------------------------------------------------------------------
%	PACKAGES AND OTHER DOCUMENT CONFIGURATIONS
%----------------------------------------------------------------------------------------

\documentclass{resume} % Use the custom resume.cls style
\usepackage{hyperref, textcomp, lipsum}
\usepackage{enumitem}
\hypersetup{
    colorlinks=true,
    linkcolor=blue,
    filecolor=blue,      
    urlcolor=blue,
}
\usepackage[left=0.5in,top=0.4in,right=0.5in,bottom=0.4in]{geometry} % 

\name{Bhanu Teja Gullapalli}
\address{\textit{\href{mailto:bgullapalli@cs.umass.edu}{bgullapalli@cs.umass.edu} \\ \href{https://www.linkedin.com/in/bhanu-teja-gullapalli/}{LinkedIn} \\  \href{https://bhanutejagullapalli.github.io}{Webpage}\footnotemark[1]}} % Your phone number and email

\begin{document}
\footnotetext[1]{Use URL \href{https://bhanutejagullapalli.github.io}{https://bhanutejagullapalli.github.io} in case hyperlinks don't work}
%----------------------------------------------------------------------------------------
%	EDUCATION SECTION
%----------------------------------------------------------------------------------------
\vspace*{-1mm}
\begin{rSection}{Education}
{\bf University of Massachusetts, Amherst} \hfill {Sept '18 - Present} \\ \textit{PhD} in Computer Science (advised by \textit{\href{http://www.tauhidurrahman.com/}{Prof. Tauhidur Rahman}}) \\\\
{\bf University of Massachusetts, Amherst} \hfill {Feb '17 - Sept '18} \\ \textit{MS} in Computer Science \hfill CGPA-3.95/4.0 \\\\
{\bf Indian Institute of Technology, Guwahati} \hfill {July '11 - May '15} \\ 
\textit{Bachelor of Technology} in  Computer Science \hfill CGPA-7.81/10.0 \\
\end{rSection}

\vspace*{-1.5mm}
\begin{rSection}{Research Interests}
\begin{itemize}[leftmargin=*]
\itemsep -0.5em 
\item Wearable Health Sensing
\item Mobile Health Systems
\item Machine Learning

% \item \textbf{K. Krishna} \& P. Jyothi - \textit{Leveraging n-grams in RNN Language Models via Loss Functions} (previously submitted to EddddddddddMNLP - 2017)
\end{itemize}

\end{rSection}
%----------------------------------------------------------------------------------------
%	WORK EXPERIENCE SECTION
%----------------------------------------------------------------------------------------
\vspace*{-1.5mm}
\begin{rSection}{Papers}

\begin{itemize}[leftmargin=*]
\item On-body Sensing of Cocaine Craving, Euphoria and Drug-Seeking Behavior Using Cardiac and Respiratory Signals \\ \textit{\textbf{Gullapalli, B.T.},  Natarajan, A., Angarita, G.A., Malison, R.T., Ganesan, D. and Rahman, T} \\ \textbf{UBICOMP 2019} 
\item A new hierarchical clustering algorithm to identify non-overlapping like-minded communities \\ \textit{Deepak, T.S., Adhya, H., Kejriwal, S., \textbf{Gullapalli, B.} and Shannigrahi, S.,} \\ \textbf{HT 16} 
\end{itemize}
\end{rSection}

\vspace*{-1.5mm}
\begin{rSection}{Key Research Projects}

\begin{rSubsection}{ Modeling Cocaine Craving, Euphoria and Drug-Seeking Behavior Using Cardiac and Respiratory Signals }{Apr '18 - Feb '19}{}{}
\item Drug addiction is a serious social and medical problem. An individual addicted to the drug follows an addiction loop which constitutes drug craving, drug-seeking behavior, drug-administration, and drug-euphoria. Existing work was only able to detect drug-administration, but this doesn't help tackle addiction or to build any intervention strategies. In this work, we show that drug-craving, euphoria and drug-seeking behavior can indeed be modeled with just ECG and breathing signal obtained from a wearable chest band. Finally, we draw various insights on the importance of features and the demographic factors which can improve the performance of the model.
\end{rSubsection}
\vspace*{-3mm}


\begin{rSubsection}{ Drug Target prediction using Deep Representation Learning}{Jan '18 - Apr '18}{  Master's Mentorship Program with IBM New York}{UMass Amherst}
\item Finding small molecules able to bind to a specific protein target is a critical aspect of drug discovery. In this project, using publicly available data on known small molecule-protein bindings from structured sources, we investigate using recently proposed deep learning representations for chemical structures and protein sequences to make drug-protein binding predictions. We propose an end-to-end model that predicts Drug-Target Interactions by taking common unprocessed representations for drugs and proteins as input. Drugs are represented as graphs, with the constituent atoms being the nodes and the bonds as edges between them. Proteins are represented as a sequence of amino acids. We apply graph convolutions on drugs and temporal convolutions on proteins to learn their fingerprint. We compare the performance networks using hand-engineered features to our end-to-end network.
\end{rSubsection}
\vspace*{-3mm}
\begin{rSubsection}{ Tree-Structured Detector Cascade}{May '17 - Aug '17}{ Summer research under \href{https://groups.cs.umass.edu/marlin/}{Benjamin Marlin}}{UMass Amherst}
\item Traditional cascading architectures are linear in general, tree-structured cascaded finds their applications in wearable technologies due to their low power consumption. In this work, we developed a novel way to grow and find the optimal configuration of a tree-structured cascade using the idea of the neutral predictor. Tested on the smoking dataset, this method increased the accuracy and F1-Score significantly.  The computational complexity of the architecture has been brought down from exponential to linear. 
\end{rSubsection}
\end{rSection}


\vspace*{-1.5mm}
\begin{rSection}{Other ACADEMIC Projects}

\begin{rSubsection}{ Improving the practical effectiveness of Exponential gap algorithm and extending it to top K arm in stochastic multi-armed bandit }{Oct'17-Dec'17}{}{}
\item Exponential gap , a best arm prediction algorithm guarantees good theoretical bounds but fails miserably in practical situations. By improving the elimination criteria, made it practically effective against current state-of-art best arm algorithms.  This algorithm is then extended to Top K arm selection by changing the sampling rate. The sample complexity of this modified algorithm is better than the current state-of-the-art Top K arm selection algorithms.
\end{rSubsection}
\vspace*{-3mm}


\begin{rSubsection}{Extending Learning-to-Optimize}{Oct'17-Dec'17}{}{}
\item  In this project, we look at the work done in trying to eliminate the hyperparameter tuning of optimization algorithms. We model this as a learning problem and train an RNN which takes gradient information as input to predict the update step of parameters. We investigate which parts of the model architecture play an important role in generalizing to different problems. We perform experiments using the MNIST dataset using different variants of the model architecture and present the performance results. 
\end{rSubsection}
\vspace*{-3mm}
\begin{rSubsection}{ Tags prediction for Stack Overflow questions using linear CRF}{Mar'17-May'17}{ }{}
\item Using Linear CRF, tags of the Stack Overflow questions are predicted. Using the idea of cascading, initially predicted tags are used as features for later tag predictions. Features extracted from the body of the post and code are given varying importance depending on the user profile. The following model gave the best accuracy so far present on this dataset. Other Features considered here include user details, question information, time posted, etc. 
\end{rSubsection}
\vspace*{-3mm}
\begin{rSubsection}{ Optical  Character  Recognition  Using  Conditional  Random  Field(CRF)}{Mar'17-Apr'17}{ }{}
\item Implemented chain structured CRFs to recognize English words from grayscale.  Implemented belief propagation for efficient inference and achieved 90\% character-level accuracy.
\end{rSubsection}
\vspace*{-3mm}
\begin{rSubsection}{ Mini Projects}{}{ }{}
\item Disease Gene identification and prioritization  using HITS and Page Rank.
\item Heart Disease Prediction Using Bayesian network.
\item  Developed a compiler for a subset of C-like programming Language (LL1 grammar) using tools from flex and bison.
\end{rSubsection}

\end{rSection}

\vspace*{-1.5mm}
\pagebreak
\begin{rSection}{Industry Experience}
\begin{rSubsection}{Software Engineer}{Jul'15 - Nov'16}{Samsung R\&D Institute, Bangalore, India}{}{}
\item Worked in the Video Editor Team of Samsung camera. Primary work was in the applications- Video Editor(Pro/Lite), Highlight player, Slow Motion. Developed and implemented theme mode in Video Editor Pro which assists the user in creating stories (Available from S8 device onwards). Developed a camera-based application which uses different composition tools to aid the user in taking good pictures.\\
\end{rSubsection}
\vspace*{-3mm}
\begin{rSubsection}{Intern}{May '14 - Jul '14}{Samsung R\&D Institute, Bangalore, India}{}{}
\item Integrated Optimized Link State Routing (OLSR) Protocol in Tizen operating system of Samsung. Added APIs which extended the functionalities of the network.
\end{rSubsection}
\end{rSection}
\vspace*{-1.5mm}
\begin{rSection}{Achievements}

\begin{itemize}[leftmargin=*]
\itemsep -0.5em 
\item Received Spot Award in Samsung R\&D Institute Bangalore for providing good solutions and coding skills
\item Won the first prize at Samsung  R\&D Institute Bangalore tech-fair for developing a location-based filter for Samsung video editor.
\item Listed among top 0.3\% students of 0.5 million appearing in\href{https://en.wikipedia.org/wiki/Joint_Entrance_Examination}{ Joint Entrance Exam, IIT-JEE }
2011
\item Secured 961 rank in All India Engineering Entrance Exam (\href{https://en.wikipedia.org/wiki/Joint_Entrance_Examination_\%E2\%80\%93_Main}{AIEEE}) 2011 taken by 1.2 million people.


\end{itemize}
\end{rSection}

\vspace*{-1mm}
\begin{rSection}{Relevant Courses}
\begin{itemize}[leftmargin=*]
\itemsep -0.5em 
\item \textbf{Machine Learning} - Machine Learning, Probabilistic Graphic Models, Theoretical Machine Learning, Artificial Intelligence, Computational Biology.
\item \textbf{Sytems} - Advanced Information Assurance, Compilers,, Computer Organization and Architecture, Operating Systems, Databases, Multimedia Systems.
\item \textbf{Theory} - Advanced Algorithms, Algorithms, Theory of computation, Structural Complexity, Discrete Maths, Data Structures. 
\item \textbf{Misc}- Optimization, Probability and Random processes, Research Methods in Empirical Computer Science.
Processes
\end{itemize}
\end{rSection}
\vspace*{-1mm}
\begin{rSection}{Technical Skills}
\begin{itemize}[leftmargin=*]
\itemsep -0.5em 
\item \textbf{Strong} - Python (with Pytorch), Java, C/C++.
\item \textbf{Familiar} - Android, SQL.
\end{itemize}
\end{rSection}
\vspace*{-1mm}


\begin{rSection}{References}
\begin{center}
\begin{tabular}{cc}
\textbf{Tauhidur Rahman} & \textbf{Deepak Ganesan} \\
Assistant Professor & Professor \\
School of Computer Science, UMass Amherst &  School of Computer Science, UMass Amherst \\
\textit{\href{https://www.cse.iitb.ac.in/~pjyothi/}{webpage} $\diamond$ \href{mailto:trahman@cs.umass.edu}{email}}  & \textit{\href{https://people.cs.umass.edu/~dganesan/}{webpage} $\diamond$ \href{mailto:dganesan@cs.umass.edu}{email}}\\
\\


\end{tabular}
\end{center}
\vspace*{-1mm}
\end{rSection}

%----------------------------------------------------------------------------------------

\end{document}
