%%%%%%%%%%%%%%%%%%%%%%%%%%%%%%%%%%%%%%%%%
% Medium Length Professional CV
% LaTeX Template
% Version 2.0 (8/5/13)
%
% This template has been downloaded from:
% http://www.LaTeXTemplates.com
%
% Original author:
% Trey Hunner (http://www.treyhunner.com/)
%
% Important note:
% This template requires the resume.cls file to be in the same directory as the
% .tex file. The resume.cls file provides the resume style used for structuring the
% document.
%
%%%%%%%%%%%%%%%%%%%%%%%%%%%%%%%%%%%%%%%%%

%----------------------------------------------------------------------------------------
%	PACKAGES AND OTHER DOCUMENT CONFIGURATIONS
%----------------------------------------------------------------------------------------

\documentclass{resume} % Use the custom resume.cls style
\usepackage{hyperref, textcomp, lipsum}
\usepackage{enumitem}
\hypersetup{
    colorlinks=true,
    linkcolor=blue,
    filecolor=blue,      
    urlcolor=blue,
}
\usepackage[left=0.5in,top=0.4in,right=0.5in,bottom=0.4in]{geometry} % 

\name{Kalpesh Krishna}
\address{\textit{\href{mailto:kalpeshk2011@gmail.com}{kalpeshk2011@gmail.com} \\ \href{https://www.linkedin.com/in/kalpesh-krishna-6b3827a6/}{LinkedIn} \\ \href{https://github.com/martiansideofthemoon}{Github} \\ \href{http://martiansideofthemoon.github.io/}{Webpage}\footnotemark[1]}} % Your phone number and email

\begin{document}
\footnotetext[1]{Use URL \href{http://martiansideofthemoon.github.io}{martiansideofthemoon.github.io} in case hyperlinks don't work}
%----------------------------------------------------------------------------------------
%	EDUCATION SECTION
%----------------------------------------------------------------------------------------
\vspace*{-1mm}
\begin{rSection}{Education}
{\bf University of Massachusetts, Amherst} \hfill {Sept '18 - Present} \\ \textit{MS/PhD} in Computer Science (advised by \textit{\href{https://people.cs.umass.edu/~miyyer/}{Prof. Mohit Iyyer}}) \\
\textbf{UMass CICS Fellowship 2018-19}\\\\
{\bf Indian Institute of Technology, Bombay} \hfill {July '14 - July '18} \\ 
\textit{B.Tech} in Electrical Engineering (\textit{Minor} in Computer Science \& Engineering)\\
(advised by \textit{\href{https://www.cse.iitb.ac.in/~pjyothi/}{Prof. Preethi Jyothi}})\\
Major GPA: \textbf{9.74/10} (\textit{$\mathbf{2^{nd}}$ among 66 students}) (Minor GPA: 10/10)\\
\textbf{Sharad Maloo Memorial Gold Medalist} for outstanding all-round performance
\end{rSection}
%----------------------------------------------------------------------------------------
%	WORK EXPERIENCE SECTION
%----------------------------------------------------------------------------------------
\vspace*{-1.5mm}
\begin{rSection}{Papers}
\begin{itemize}[leftmargin=*]
\item Generating Question-Answer Hierarchies \\ \textit{\textbf{K. Krishna}, M. Iyyer} \\ \textbf{ACL 2019} 
\item Trick or TReAT: Thematic Reinforcement for Artistic Typography \\ \textit{P. Tendulkar, \textbf{K. Krishna}, R. Selvaraju, D. Parikh} \\ \textbf{ICCC 2019} \\ \texttt{arXiv:1903.07820 [cs.CV]} 
\item Revisiting the Importance of Encoding Logic Rules in Sentiment Classification \\ \textit{\textbf{K. Krishna}, P. Jyothi, M. Iyyer} \\ \textbf{EMNLP 2018} (\textit{oral presentation, short paper}) \\
\texttt{arXiv:1808.07733 [cs.CL]}
\item Hierarchical Multitask Learning for CTC-based Speech Recognition  \\ \textit{\textbf{K. Krishna}, S. Toshniwal, K. Livescu} \\ \texttt{arXiv:1807.06234v2 [cs.CL]} 
\item A Study of All-Convolutional Encoders for Connectionist Temporal Classification\\ \textit{\textbf{K. Krishna}, L. Lu, K. Gimpel,  K. Livescu}\\ \textbf{ICASSP 2018} (\textit{Awarded SPS Travel Grant}) \\ \texttt{arXiv:1710.10398v2 [cs.CL]} 
% \item \textbf{K. Krishna} \& P. Jyothi - \textit{Leveraging n-grams in RNN Language Models via Loss Functions} (previously submitted to EddddddddddMNLP - 2017)
\end{itemize}
\end{rSection}
\vspace*{-1.5mm}
\begin{rSection}{Experience}
{\bf Toyota Technological Institute at Chicago}{ \hfill May '17 - July '17}\\ \textit{Visiting Student under \href{http://ttic.uchicago.edu/~klivescu/}{Karen Livescu}, \href{http://ttic.uchicago.edu/~llu/}{Liang Lu} and \href{http://ttic.uchicago.edu/~kgimpel/}{Kevin Gimpel}}{\hfill Chicago, IL}

{\bf Mozilla Foundation}{\hfill May '16 - August '16} \\ \textit{Google Summer of Code student under \href{https://github.com/armenzg}{Armen Zambrano}}{ \hfill Mumbai, India / London, UK}
\end{rSection}
\vspace*{-1.5mm}
\begin{rSection}{Scholastic Achievements}
\begin{itemize}[leftmargin=*]
\itemsep -0.5em 
\item \textbf{Cargill Global Scholar} (2016-18) Selected by the \href{https://en.wikipedia.org/wiki/Institute_of_International_Education}{International Institute of Education} and \href{https://en.wikipedia.org/wiki/Cargill}{Cargill} (largest private corporation in USA) for a global leadership \href{https://www.cargillglobalscholars.com/}{program}.
\item Received the Institute Academic Prize for standing $2^{nd}$ in the  sophomore year 2015-16.
\item Selected for \href{https://www.lti.cs.cmu.edu/2017-jsalt-undergraduate}{JSALT '17}, organized by JHU's \href{https://www.clsp.jhu.edu/}{Center for Language and Speech Processing}\footnotemark[2]. \footnotetext[2]{Couldn't attend due to clashing college schedule}
\item \textbf{Top 10} at the Astronomy Olympiad's Indian Selection Camp for IOAA '14, ($\sim$20000 applicants).
\item All India Rank 2 \textit{(out of 132k)} in \href{https://en.wikipedia.org/wiki/Indian_Certificate_of_Secondary_Education}{ICSE} '12, All India Rank 93 in \href{https://en.wikipedia.org/wiki/Joint_Entrance_Examination}{JEE Advanced} '14 \textit{(out of 126k)} and All India Rank 34 in \href{https://en.wikipedia.org/wiki/Joint_Entrance_Examination}{JEE Mains} '14 \textit{(out of 1.4M candidates)}.
\end{itemize}
\end{rSection}
\vspace*{-1.5mm}
\begin{rSection}{Responsibilities}
\begin{itemize}[leftmargin=*]
\itemsep -0.5em 
\item \textbf{Manager, Web and Coding Club} at IIT Bombay (2016-17)\\ \textit{(awarded Institute Organizational Color 2016-17)}
\item \textbf{Institute Student Mentor} at IIT Bombay (2017-18)
\item \textbf{Teaching Assistant} at IIT Bombay in \textit{Computer Programming} (2016) and \textit{Linear Algebra} (2017)
\end{itemize}
\end{rSection}
%----------------------------------------------------------------------------------------

\end{document}
